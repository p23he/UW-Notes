\documentclass[12pt]{article}
%\usepackage{tasks}
\usepackage{exsheets}
\usepackage{fullpage}
\usepackage{multicol}
\usepackage{amsmath,amsthm,amssymb}
\SetupExSheets[question]{type=exam}

\newtheorem{theorem}{Theorem}
\newtheorem{corollary}{Corollary}[theorem]
\newtheorem{lemma}[theorem]{Lemma}
\newtheorem{proposition}[theorem]{Proposition}

\newcommand{\N}{\mathbb{N}}
\newcommand{\Z}{\mathbb{Z}}
\newcommand{\R}{\mathbb{R}}
\title{MATH237 Lecture 02}
\author{Peter He}
\date{January 8 2020}

\begin{document}

\maketitle

\section{Scalar functions and graphs continued}
\subsection{Example of graphing: $z=2x+y$}

Recall that we should sketch the level curves and cross-sections, then consider the symmetry.
\begin{itemize}
    \item If $z=c$ for some constant $c$, we have the line $y=-2x+c$. So our level curves will look like lines with negative slope.
    \item Consider the cross sections $z=2x+c$ and $z=2c+y$. The cross sections are also lines (on the $z-x$ and $z-y$ plane). They are also lines, both with positive slope, with the first one being steeper than the other.
    \item Now we sketch the entire graph.
\end{itemize}

\subsection{Example 2: $z=x^2+y^2$}
\begin{itemize}
    \item Level curves: $c=x^2+y^2$ - circle
    \item Cross Sections: $z=c^2+y^2$ and $z=x^2+c^2$ - parabola
    \item Now we sketch the entire graph - it's a parabolic cone, infinite paraboloid
\end{itemize}

\subsection{Example 3: $z=\sqrt{x^2+y^2}$}
\begin{itemize}
    \item Level curves: $c=\sqrt{x^2+y^2}=c^2+x^2+y^2$ - circle again
    \item Cross Sections: $z^2 = x^2+c^2$ - what shape is this? If $c=0$ then the graph looks like $z=|x|$ on the $z-x$ plane.
    \item The shape is a cone - hyperbola
\end{itemize}

\subsection{Example 4: $z=xy$}
\begin{itemize}
    \item Level curves: $c=xy$ - shape of the reciprocal function 
    \item Cross Sections: $z=cx$ and $z=cy$ - lines with different slopes
\end{itemize}

\subsection{Matching with images from the slide}
\begin{itemize}
    \item Figure 1 corresponds to example 4
    \item Figure 2 corresponds to example 1
    \item Figure 3 corresponds to example 3
    \item Figure 4 corresponds to example 2
\end{itemize}

\section{Limits}
\textbf{Definition:} A function $f:\R\to\R$ is \textbf{continuous} at $a$ if $\lim\limits_{x\to a} f(x)=f(a)$.\\
\textbf{Definition:} A function $f:\R^2\to\R$ is \textbf{continuous} at $(a,b)$ if $\lim\limits_{(x,y)\to (a,b)}f(x,y)=f(a,b)$.\\
What does this mean? For a function $f:\R\to R$, $\lim\limits_{x\to a}f(x)=L$ means as $x$ gets close to $a$, $f(x)$ gets close to $L$. More precisely, for every interval around $L$, we can construct an interval around $a$ such that $f$ maps every element in that interval to $L's$ interval.\\\\
How about for $f:\R^2\to\R$?. \\
\textbf{Definition:} $\lim_{(x,y)\to (a,b)}f(x,y)=L$ means that for any $\varepsilon>0$, there is a $\delta>0$ such that if $||\underline x-\underline a||<\delta$ then $|f(\underline x)-L|<\varepsilon$.

\subsection{Example}
\[f(x,y)=\begin{cases}
    \frac{\sin xy}{x^2+y^2} &(x,y)\neq (0,0)\\
    0 &(x,y)=(0,0)
\end{cases}
    \]
Does \[\lim_{\underline x\to \underline 0}f(\underline x)\]
exist?
\end{document}