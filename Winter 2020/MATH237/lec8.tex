\documentclass[12pt]{article}
%\usepackage{tasks}
\usepackage{exsheets}
\usepackage{fullpage}
\usepackage{multicol}
\usepackage{amsmath,amsthm,amssymb}
\SetupExSheets[question]{type=exam}

\newtheorem{theorem}{Theorem}
\newtheorem{corollary}{Corollary}[theorem]
\newtheorem{lemma}[theorem]{Lemma}
\newtheorem{proposition}[theorem]{Proposition}
\theoremstyle{definition}
\newtheorem{definition}{Definition}[section]

\newcommand{\N}{\mathbb{N}}
\newcommand{\Z}{\mathbb{Z}}
\newcommand{\R}{\mathbb{R}}
\title{MATH237 Lecture 08}
\author{Peter He}
\date{January 22 2020}

\begin{document}

\maketitle

\section{Clicker Review!}
\[f(x,y)=\begin{cases}\frac{xy}{\sqrt{x^2+y^2}}&(x,y)\neq(0,0)\\0&(x,y)=(0,0)\end{cases}\]
Does $L(x,y)$ exist at $(0,0)$? Yes, $L(x,y)=0$.

\section{Rigourous Definition of Differentiability}

\begin{definition}
    A function $f:B(r,\underline a)\subset\R^2\to\R, r>0$ is \textbf{differentiable} at $\underline a=(a,b)$ if he partial derviatives exist at $\underline a$ and the linear approximation \[L_{\underline a}(x)=f(\underline a)+f_x(\underline a)(x-a)+f_y(\underline a)(y-b)\] satisifies \[\lim_{\underline x\to\underline a}\frac{|f(x)-L_{\underline a}(\underline x)|}{||\underline a-\underline a||}\]
\end{definition}

\subsection{Example}
Let \[f(x,y)=x^2+y^2\] and \[\underline a=(1,2)\] Is $f$ differentiable at $(1,2)$? Yes.\\\\
Solution:\\
Calculate $L(x,y)$ at $(1,2)$. 

\begin{align*}
    f(1,2)&=5\\\\
    f_x(x,y)&=2x\implies f_x(1,2)=2\\
    f_y(x,y)&=2y\implies f_y(1,2)=4\\\\
    L(x,y)&=-5+2x+4y
\end{align*}
Now we need to show \[\lim_{\underline x\to\underline a}\frac{|f(x,y)-L(x,y)|}{||(x,y)-(1,2)||}=0\] so 
\begin{align*}
    \frac{|f(x,y)-L(x,y)|}{||(x,y)-(1,2)||}&=\frac{|x^2+y^2-5-2(x-1)-4(y-2)|}{\sqrt{(x-1)^2+(y-2)^2}}\\
    &=\sqrt{(x-1)^2}
\end{align*}

\end{document}