\documentclass[12pt]{article}
%\usepackage{tasks}
\usepackage{exsheets}
\usepackage{fullpage}
\usepackage{multicol}
\usepackage{amsmath,amsthm,amssymb}
\SetupExSheets[question]{type=exam}

\newtheorem{theorem}{Theorem}
\newtheorem{corollary}{Corollary}[theorem]
\newtheorem{lemma}[theorem]{Lemma}
\newtheorem{proposition}[theorem]{Proposition}

\newcommand{\N}{\mathbb{N}}
\newcommand{\Z}{\mathbb{Z}}
\newcommand{\R}{\mathbb{R}}
\title{MATH237 Lecture 01}
\author{Peter He}
\date{January 6 2020}

\begin{document}

\maketitle

\section{Overview of the Course}
We covered functions of 1 variable, in general $f:\R\to\R$. We covered topics such as
\begin{itemize}
    \item Limits
    \item Continuity
    \item Linear approximation
    \item Differentiation
    \item Integration
\end{itemize}
We will now cover funcctions of $>1$ variables, in general $f:\R^n\to\R^m$. Examples of multivariable functions:
\begin{itemize}
    \item Ocean temp
    \item Temp of Canada
    \item wind in Canada
    \item factory production
\end{itemize}
Scalar functions ($f:\R^n\to\R$)
\begin{itemize}
    \item Scalar functions are the focus of this course
    \item most of the discussion is for $n=2$
    \item Generalization to $n=$ and arbitrary $n$
\end{itemize}

\section{Graphs}
We will discuss \textbf{terminology} and \textbf{visualization}\\\\
Review:\\
A function $f:A\to B$ is a rule that associates each $a\in A$ to a unique element of $B$, $f(a)$. The domain of $f$ is $A$, $D(f)$. The range of $f$ is $R(f)=\{b\in B|b=f(a), a\in A\}$. $f(x,y)$ can mean the value of $f$ at the point $(x,y)$ or more usually that $f$ is a function of 2 variables.\\\\
Example: $f(x,y)=x^2+y^2$. We can see that $f:\R^2\to \R, D(f)=\R^2,$ and $R(f)=\{z\in \R|z\geq 0\}\subset\R$

\subsection{Visualization of $f:\R\to\R$}
Methods:
\begin{itemize}
    \item Level curves
    \item Cross-sections
    \item Symmetry
    \item Analysis
    \item Computer plots
\end{itemize}
Example: $f(x,y)=z$.
\begin{itemize}
    \item Level curves: $z=c$: $x^2+y^2=c$ for some constant $c$ (we would get some sort of circle, analogous to looking "down" on the shape)
    \item Cross-sections: $y=c$ or $x=c$: $x^2+c^2=z$ (we would get some sort of parabola)
\end{itemize}
We might guess that the shape is a parabolic cone thing.

Let's look at $f(x,y)=\sin(x^2+y^2)$. Level curves are of the form $c=\sin(x^2+y^2)$ or $\arcsin c=x^2+y^2$. We expect to see a bunch of circles when graphed.\\
Cross sections: $z=\sin(x^2+c^2)$. We expect to see a distorted sine curve shifted left.\\
Symmetry:

\end{document}