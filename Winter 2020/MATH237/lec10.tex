\documentclass[12pt]{article}
%\usepackage{tasks}
\usepackage{exsheets}
\usepackage{fullpage}
\usepackage{multicol}
\usepackage{amsmath,amsthm,amssymb}
\SetupExSheets[question]{type=exam}

\newtheorem{theorem}{Theorem}
\newtheorem{corollary}{Corollary}[theorem]
\newtheorem{lemma}[theorem]{Lemma}
\newtheorem{proposition}[theorem]{Proposition}
\theoremstyle{definition}
\newtheorem{definition}{Definition}[section]

\newcommand{\N}{\mathbb{N}}
\newcommand{\Z}{\mathbb{Z}}
\newcommand{\R}{\mathbb{R}}
\title{MATH237 Lecture 10}
\author{Peter He}
\date{January 27 2020}

\begin{document}

\maketitle

\section{Last Class}
Recall the defintion of a function being differentiable.
\begin{definition}
    A function $f:B(r,\underline a)\subset \R^2\to\R, r>0$ is differentiable at $\underline a=(a,b)$ if the partial derivatives exist at $\underline a$ and the linear approximation \[L_{\underline a}(\underline x)=f(\underline a)+f_x(\underline a)(x-a)+f_y(\underline a)(y-b)\] satisfies \[\lim_{\underline x\to \underline a}\frac{|f(\underline x)-L_{\underline a}(\underline x)|}{||\underline x-\underline a||}\]
\end{definition}
Intuitively, being differentiable exactly means having a good linear approximation at the point.
\section{More on Differentiability}
\begin{proposition}
    Consider $f:B(r,\underline a)\subset\R^n\to\R, r>0$. If all the partial derivatives exist within $B$ and are continuous at $\underline a$, then $f$ is differentiable at $\underline a$.
\end{proposition}
\begin{theorem}[Mean Value Theorem on $\R$]
    Consider $g[c,d]\to\R$. If $f$ is continuous on $[c,d]$ and differentiable on $(c,d)$ then there is a point $x_0$ such that \[g(c)-g(d)=g'(x_0)(c-d)\]
\end{theorem}

\begin{proof}
    Proof of the claim on slides.
\end{proof}

\subsection{Example 1}
Consider \[f(x,y)=\begin{cases}(x^2+y^2)\sin(\frac{1}{\sqrt{x^2+y^2}})&(x,y)\neq(0,0)\\0&(x,y)=(0,0)\end{cases}\]
Where is $f$ differentiable?\\\\
\underline{At $(x,y)\neq (0,0)$}: Is $f$ differentiable? (Clicker: Yes or No$\to$ Yes)\\
Compute the partial derivatives, and conclude that they are continuous for non 0 points.
\underline{At $(x,y)=(0,0)$}:\\
\begin{align*}
    f_x?:\lim_{h\to 0}\frac{f(h+0,0)-f(0,0)}{h}&=\lim_{h\to 0}\frac{h^2\sin(\frac{1}{|h|})}{h}\\
    &=\lim_{h\to 0}h\sin(\frac{1}{|h|})\\
    &=0
\end{align*}
Thus $f_x(0,0)$ exists and $f_x(0,0)=0$. Similarly, $f_y(0,0)=0$. Thus, \[L(x,y)=f(0,0)+f_x(0,0)(x-0)+f_y(0,0)(y-0)=0\]
Check differentiability using the definition:
\begin{align*}
    \frac{|f(x,y)-L(x,y)|}{\sqrt{x^2+y^2}}&=\frac{|(x^2+y^2)\sin(\frac{1}{\sqrt{x^2+y^2}})-0|}{\sqrt{x^2+y^2}}\\
    &=\sqrt{x^2+y^2}|\sin(\frac{1}{\sqrt{x^2+y^2}})|\\
    &\leq \sqrt{x^2+y^2}
\end{align*}
So \[\lim_{||\underline x-\underline 0}\frac{|f(x,y)-L(x,y)|}{\sqrt{x^2+y^2}}=0\] by the Squeeze Theorem. Therefore, $f$ is differentiable on $\R^2$.
\subsubsection{Is $f_x$ continuous at $(0,0)$?}
$f_x(x,y)=2x\sin(\frac{1}{\sqrt{x^2+y^2}})-\frac{x}{\sqrt{x^2+y^2}}\cos(\frac{1}{\sqrt{x^2+y^2}})$. The limit of the first term is 0 as $\underline x\to \underline 0$.
As for the second term, the limit does not exist. So use the definition for niche/edge case points.

\subsection{Relationships between stuff}

\begin{itemize}
    \item Differentiability implies existence of partial derivatives AND continuity
    \item Continuous partial derivatives implies differentiability
\end{itemize}
\end{document}