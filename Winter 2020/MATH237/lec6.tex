\documentclass[12pt]{article}
%\usepackage{tasks}
\usepackage{exsheets}
\usepackage{fullpage}
\usepackage{multicol}
\usepackage{amsmath,amsthm,amssymb}
\SetupExSheets[question]{type=exam}

\newtheorem{theorem}{Theorem}
\newtheorem{corollary}{Corollary}[theorem]
\newtheorem{lemma}[theorem]{Lemma}
\newtheorem{proposition}[theorem]{Proposition}
\theoremstyle{definition}
\newtheorem{definition}{Definition}[section]

\newcommand{\N}{\mathbb{N}}
\newcommand{\Z}{\mathbb{Z}}
\newcommand{\R}{\mathbb{R}}
\title{MATH237 Lecture 06}
\author{Peter He}
\date{January 17 2020}

\begin{document}

\maketitle

\section{Clicker!}
Is \[f(x,y)=\begin{cases}\frac{xy^2}{x^2+y^6}&(x,y)\neq(0,0)\\0&(x,y)=(0,0)\end{cases}\] continuous at $(0,0)$?

Yes! We check if the limit exists at (0,0) and if it does, check if it's 0. We will try $y=mx$.
\begin{align*}
    \lim_{x\to 0}f(x,mx)&=\lim_{x\to 0}\frac{m^4x^5}{x^2+m^6x^6}\\
    &=\lim_{x\to 0}\frac{m^4x^3}{1+m^6x^4}\\
    &=0
\end{align*}
we suspect the limit is 0. We prove it 
\begin{proof}
    \begin{align*}
        0\leq \left|\frac{xy^2}{x^2+y^6} -0\right|&\leq \frac{|x|y^4}{2|x||y|^3}\qquad\text{Young's Inequality}\\
        &=\frac{1}{2}|y|
    \end{align*}
    Since $\lim_{\underline x\to\underline 0}\frac{1}{2}|y|=0, \lim_{\underline x\to\underline 0}\frac{xy^4}{x^2+y^6}=0=f(0,0)$. Thus the function is continuous at $(0,0)$.
\end{proof}

\section{Partial Derivatives}
\begin{definition}[Partial Derivative]
    Let $f:B(r,\underline a)\subset \R^2\to\R$. The \textit{partial derivative of $f(x,y)$ with respect to $x$} at $\underline a=(a,b)$ is (if the limit exists)
    \[\frac{\partial f}{\partial x}(a,b)=\lim_{t\to 0}\frac{f(a+t,b)-f(a,b)}{t}\]
    and the \textit{partial derivative of $f(x,y)$ with respect to $y$} at $\underline a=(a,b)$ is (if the limit exists)
    \[\frac{\partial f}{\partial x}(a,b)=\lim_{t\to 0}\frac{f(a,b+t)-f(a,b)}{t}\]
\end{definition}

\subsection{Notation}
\begin{itemize}
    \item point in $\R^n$ indicated by underscore (class) or bold (text): $\underline a$ or $\mathbf a$
    \item partial derivative of $f(x,y)$ with respect to $x$ indicated by: \[\frac{\partial f}{\partial x}, f_x, D_1f, D_x\]
\end{itemize}

\begin{definition}
    Suppose that $f:B(r,\underline a)\subset \R^n\to \R$ has partial dervatives $\underline a$. The \textit{gradient} of $f$ at $\underline a$ is \[\nabla f(\underline a)=(f_{x_1}(\underline a),\dots, f_{x_n}(\underline a))\]
\end{definition}

\subsection{Example 1}
Let \[f(x,y)=x^2+xy^3\] Then we have
\begin{align*}
    \lim_{t\to 0}\frac{f(x+t), y-f(x,y)}{t}&=\lim_{t\to 0}\frac{(x+t)^2+(x+t)y^3-x^2-xy^3}{t}\\
    &=\lim_{t\to 0}\frac{2xt+t^2+ty^3}{t}\\
    &=2x+y^3
\end{align*}

Thus $\frac{\partial f}{\partial x}(x,y)=2x+y^3$. Or,

We can also use regular rules if we treat $y$ is constant.

\subsection{Example 2}
Let \[f(x,y)=\sqrt{x^2+y^2}=(x^2+y^2)^{1/2}, (x,y)\]

\begin{align*}
    \frac{\partial f}{\partial x}&=\frac{x}{\sqrt{x^2+y^2}}\\
    \frac{\partial f}{\partial y}&=\frac{y}{\sqrt{x^2+y^2}}
\end{align*}
at $(x,y)\neq(0,0)$. For $(x,y)=(0,0)$, we must use the definition and check if the limit even exists:
\begin{align*}
    \lim_{t\to 0}\frac{f(0+t, 0)-f(0,0)}{t}&=\lim_{t\to 0}\frac{\sqrt{t^2}-0}{t}\\
    &=\lim_{t\to 0}\frac{|t|}{t}
\end{align*}
The limit does not exist so $\frac{\partial f}{\partial x}(0,0)$ doesn't exist. Similarly, $\frac{\partial f}{\partial y}(0,0)$ doesn't exist. $f$ is continuous, so this example shows that continuity does not imply existence of partial derivative.


\subsection{Example 3}
Let \[f(x,y)=\begin{cases}\frac{xy}{x^2+y^2}&(x,y)\neq(0,0)\\ 0&(x,y)=(0,0)\end{cases}\]
Then for $(x,y)\neq (0,0)$:
\begin{align*}
    \frac{\partial f}{\partial x}=\frac{y(x^2+y^2)-xy2x}{(x^2+y^2)^2}
\end{align*}
At $(x,y)=(0,0)$:
\begin{align*}
    \lim_{t\to 0}\frac{f(t+0,0)-f(0,0)}{t}&=\lim_{t\to 0}\frac{0-0}{t}\\
    &=0
\end{align*}
So $\frac{\partial f}{\partial x}(0,0)=0$. Similarly, by symmetry, $\frac{\partial f}{\partial y}(0,0)=0$. It can be shown that $\lim_{\underline x\to\underline 0}f(\underline x)$ doesn't exist, so the function is not continuous. This example shows that even if all the partial derivatives exist, the function may not be continuous.

\subsection{Remarks}
\begin{itemize}
    \item Calculate a partial derivative just as you would for single variable calculus: all other variables are held constant
    \item Examples illustrate that:
    \begin{itemize}
        \item A continuous function may fail to have partial derivatives
        \item A discontinuous function may have partial derivatives
    \end{itemize}
\end{itemize}


\end{document}