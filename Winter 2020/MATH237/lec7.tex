\documentclass[12pt]{article}
%\usepackage{tasks}
\usepackage{exsheets}
\usepackage{fullpage}
\usepackage{multicol}
\usepackage{amsmath,amsthm,amssymb}
\SetupExSheets[question]{type=exam}

\newtheorem{theorem}{Theorem}
\newtheorem{corollary}{Corollary}[theorem]
\newtheorem{lemma}[theorem]{Lemma}
\newtheorem{proposition}[theorem]{Proposition}
\theoremstyle{definition}
\newtheorem{definition}{Definition}[section]

\newcommand{\N}{\mathbb{N}}
\newcommand{\Z}{\mathbb{Z}}
\newcommand{\R}{\mathbb{R}}
\title{MATH237 Lecture 07}
\author{Peter He}
\date{January 20 2020}

\begin{document}

\maketitle

\section{Last Class}
Clicker!: What are the partial derivatives of $f(x,y)=e^{x^2y}\cos(x)+y$?

\subsection{Second Partial Derivatives}
The partial derivatives of the (first) partial derivatives of $f$ are called the \underline{second partial derivatives} of $f$. Notation:
\[\frac{\partial^2f}{\partial x^2}=\frac{\partial}{\partial x}\left(\frac{\partial f}{\partial x}\right)=f_{xx}=D^2_xf\]

\subsection{Example 1}
\[f(x,y)=x^2+xy^3\]
\[\frac{\partial f}{\partial x}(x,y)=2x+y^3, \frac{\partial f}{\partial y}(x,y)=3xy^2\]
\[\frac{\partial^2f}{\partial x^2}(x,y)=2, \frac{\partial f^2}{\partial y\partial x}(x,y)=3y^2\]
\[\frac{\partial^2f}{\partial x\partial y}(x,y)=3y^2\]

Consider $f:B(r,\underline a)\subset \R^2\to \R$. If $f_{xy}$ and $f_{yx}$ are defined on $B(r,\underline a)$ and continuous at $\underline a$, then $f_{ya}(\underline a)=f_{xy}(\underline a)$

\section{Linear Approximation}
For functions $f:\R\to \R$, the tangent line \[L_a(x)=f(a)+f'(a)(x-a)\] can be used to approximate $f$ when $f'(a)$ is defined.\\\\
For $f:\R^2\to \R$, a linear approximation at $(a,b)$ would have the form \[L_{(a,b)}(x,y)+A+B(x-a)+C(y-b)\]
What should our constants be?

\[L_{(a,b)}(a,b)=f(a,b)\implies A=f(a,b)\]
\begin{align*}
    \lim_{t\to\underline a}\frac{L_{(a,b)}(a+t, b)-L_{(a,b)}(a,b)}{t}&=\lim_{t\to \underline a}\frac{A+Bt+0-A}{t}\\
    &= B\implies B=f_x(a,b)
\end{align*}


Similarly, $C=f_y(a,b)$.

Thus,

\begin{definition}
    Consider $f:B(r,\underline a)\subset \R^2\to \R$ and assume all partial derivatives of $f$ exist at $\underline a$. The \textbf{linearization} or \textbf{lienar approximation} of $f$ at $\underline a$ implies\[L_{\underline a}(x)=f(\underline a)+\nabla f(a)\cdot(\underline x-\underline a)\]
\end{definition}

\subsection{Increment form of linear approximation}

\[f(x,y)\approx L_{(a,b)}(x,y)\]
\begin{align*}
    L_{(a,b)}(x,y)&=f(a,b)+f_x(a,b)(x-a)+f_y(a,b)(y-b)\\
    L_{(a,b)}(x,y)-f(a,b)&=f_x(a,b)(x-a)+f_y(a,b)(y-b)
\end{align*}

\subsection{Example 1}
Consider \[f(x,y)=x^2+y^2\]
Calculate $L(x,y)$ at $(1,2):L_{(1,2)}(x,y)$.\\\\
\begin{align*}
    f(1,2)&=1+4=5\\\\
    f_x(x,y)&=2x, f_x(1,2)=2\\
    f_y(x,y)&=2y, f_y(1,2)=4\\\\
    L(x,y)&=f(1,2)+f_x(1,2)(x-1)+f_y(1,2)(y-2)\\
    &=5+2(x-1)+4(y-2)
\end{align*}
Comparision:
\begin{tabular}{c|c|c}
    point & $L$ & $f$\\\hline
    (1.1, 2.1) & 5.6 & 5.62\\
    (.9, 2) & 4.8 & 4.81\\
    (1, 1.9) & \\
\end{tabular}

\subsection{Example 2}
\[f(x,y)=\sqrt{|xy|}\]
Use linear approximation to approximate $f$ near $(0,0)$.

\begin{align*}
    f(0,0)=0\\\\
    f_x:\frac{\partial}{\partial x}(|xy|^{1/2})=\frac{1}{2}|xy|^{-1/2}\frac{\partial}{\partial x}|xy|\text{ Not valid!}\\\\
    \lim_{t\to 0}\frac{f(t+0, 0)-f(0,0)}{t}=\lim_{t\to 0}\frac{0-0}{t}=0\\\\
    \text{Similarly, }\frac{\partial f}{\partial y}(0,0)=0
\end{align*}
So, $L(x,y)=0$.

\subsection{Example 3}
\[f(x,y)=|xy|\]
Calculate $L(x,y)$ at $(0,0)$.
\begin{align*}
    f(0,0)&=0\\\\
    f_x(0,0)&=\lim_{t\to 0}\frac{f(t+0, 0)-f(0,0)}{t}&=\lim_{t\to 0}\frac{0-0}{t}=0\\
    f_y(0,0)&=0\text{ by similar analysis}\\\\
    L_{(0,0)}(x,y)&=f(0,0)+f_x(0,0)(x-0)+f_y(0,0)(y-0)=0
\end{align*}
Comparision:
\begin{tabular}{c|c|c}
    point & $L$ & $f$\\\hline 
    (.1, -.1) & 0 & .01\\
    (.01, .1) & 0 & .001
\end{tabular}

\subsection{Example 4}
Same $f$ as before but at $(1,2)$.

\begin{align*}
    f(0,0)&=2\\
    f_x:f(x,y)&=xy\qquad\text{near }(1,2)\implies f_x(1,2)=2\\
    f_y:f(x,y)&=x\implies f_y(1,2)=1\\
    L_{(1,2)}(x,y)&=2+2(x-1)+(y-2)
\end{align*}    
\end{document}